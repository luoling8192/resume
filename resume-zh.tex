% Chinese version
\documentclass[zh]{resume}

% Adjust icon size(default: same size as the text)
\iconsize{\Large}

% File information shown at the footer of the last page
\fileinfo{%
  \faCopyright{} 2025, RainbowBird \hspace{0.5em}
  \creativecommons{by}{3.0} \hspace{0.5em}
  \faEdit{} \today
}

\name{RainbowBird}{洛灵}
\tagline{Go 开发工程师 | 全栈开发工程师}
\profile{
  \email{rbxin2003@outlook.com}
  \github{luoling8192} \\
  \address{远程}
  \birthday{2003-11}
  \icontext{\faWeixin}{@luoling8192}
}

\begin{document}
\makeheader

%======================================================================
% Summary & Objectives
%======================================================================
\sectionTitle{简介}{\faInfo}
{\onehalfspacing
\hspace{2em} 全栈工程师,从前端开发起步,专注\textbf{云原生}和\textbf{Web3 基础设施}领域。深入掌握\textbf{Go 微服务开发}、\textbf{Kubernetes 集群管理}以及\textbf{区块链节点运维},同时对 LLM 与 AI Agent 等前沿技术有深入研究。
\par}
{\onehalfspacing
\hspace{2em} 具备端到端的全链路开发能力,能独立完成从前端界面到后端微服务的完整系统设计。专长于\textbf{高性能 Go 服务开发}、\textbf{云原生架构设计}和\textbf{Web3 基础设施建设}。在多个项目中负责微服务架构设计、高可用性方案制定、数据持久化方案选型,以及基于云服务的成本优化工作。
\par}
{\onehalfspacing
\hspace{2em} 在 DevOps 实践中,设计并实现了基于 Kubernetes 的\textbf{区块链节点管理平台},支持多链节点的自动化部署与运维,实现系统可用性 99.99\%。同时构建了完整的 CI/CD 流水线和监控告警体系,确保服务稳定性。
\par}
{\onehalfspacing
\hspace{2em} 热衷技术创新与实践,在 Web3、云原生和 AI 领域持续探索。善于快速掌握并整合新技术,致力于打造高可靠、可扩展的企业级解决方案。
\par}

%======================================================================
\sectionTitle{技能}{\faWrench}
%======================================================================
\begin{competences}
  \comptence{\textbf{后端开发}}{
    Go (1.20+) \quad
    微服务架构 \quad
    gRPC / REST API \quad
    分布式系统设计
  }
  \comptence{\textbf{技术栈}}{
    Echo · Fx · Ent \quad
    Kafka · Redis \quad
    PostgreSQL \quad
    TypeScript · Python
  }
  \comptence{\textbf{云原生}}{
    Kubernetes \quad
    Docker \quad
    AWS \quad
    CI/CD
  }
  \comptence{\textbf{可观测性}}{
    Prometheus \quad
    Grafana \quad
    OpenTelemetry \quad
    ELK
  }
  \comptence{\textbf{全栈开发}}{
    Vue3 / React \quad
    Vite · Pinia \quad
    Serverless \quad
    Monorepo
  }
\end{competences}

%======================================================================
\sectionTitle{工作经历}{\faBriefcase}
%======================================================================
\begin{experiences}
  \experience%
    [2024-06]%
    {至今}%
    {Go 高级开发工程师 @ 八位猫(上海)}%
    [\begin{itemize}
      \item \textbf{区块链基础设施}
        \begin{itemize}
          \item 主导区块链基础设施建设,基于 Kubernetes 构建高可用多链节点集群,支持 Ethereum、Astar 等主流公链,通过多可用区部署与自动故障转移,实现系统可用性 99.99\%。
          \item 设计三层立体化告警机制,包括服务主动告警、日志智能分析及 K8s 健康检查,配合自动化运维脚本,将故障平均恢复时间(MTTR)优化至 5 分钟以内。
          \item 优化 gRPC 工作流,设计基于 GitHub Actions 的 Protobuf 自动化生成与同步方案,显著提升团队协作效率,降低接口维护成本。
          \item 推行测试驱动开发(TDD),构建完整测试工具链,包括数据库与缓存 Mock 框架,实现核心业务 95\% 测试覆盖率,生产环境故障率降低 70\%。
        \end{itemize}
      \item \textbf{区块链交易系统}
        \begin{itemize}
          \item 设计高性能分布式队列中间件,基于 Fx 依赖注入与 Echo 框架,结合 Kafka 消息队列和 Redis 分布式缓存,实现可靠的事件处理,支撑日均百万级交易规模。
          \item 实现智能数据分片与多级缓存策略,优化查询性能提升 5 倍,将平均响应时间控制在 50ms 以内,大幅提升用户体验。
          \item 深度整合 OpenTelemetry 可观测性框架,实现全链路追踪与性能监控,配合 Swagger 自动化文档,研发效率提升 60\%。
          \item 构建端到端 DevOps 系统,实现从代码提交到生产部署的全流程自动化,通过金丝雀发布与自动回滚确保系统稳定性达 99.9\%。
        \end{itemize}
      \end{itemize}]
\end{experiences}

%======================================================================
\sectionTitle{项目经历}{\faBriefcase}
%======================================================================
\begin{projects}
  \project{2024-07}[2024-12]{后端负责人}{\link{https://guii.ai}{https://guii.ai}}{Guii.AI}{
    \begin{itemize}
      % \item 一款 AI 驱动前端开发工具,通过自然语言交互实现界面生成与优化,打造从原型设计到细节调优的一站式开发体验。
      \item 设计 Agent 编排引擎,通过参数校验与结构化处理,实现了 Function Calling 能力。
      \item 实现了项目组件的智能识别与文档抓取,构建组件知识库实现精准代码生成。
      \item 构建企业级工作流引擎,基于 Redis 实现高可靠消息通信,配备完整的断线重连、消息持久化方案,通过任务编排与状态追踪将系统吞吐量提升至 10k QPS。
    \end{itemize}
  }[HonoJS, Supabase, PostgreSQL, Redis, Drizzle]
  \separator{0.5ex}
  \project{2024-02}[2024-05]{前端负责人}{\link{https://portforward.zeroteam.top}{https://portforward.zeroteam.top}}{PortForwardGo}{
    \begin{itemize}
      \item 基于 Vue3 构建现代化前端架构,采用组件化开发提升代码复用率,实现多渠道支付与服务器状态实时监控等核心功能。
      \item 设计低代码表单引擎,支持后端动态表单生成与验证,通过可配置的组件库与数据管理接口,将表单开发效率提升 70\%。
      \item 优化性能体验,引入 SSR 技术将首屏加载时间降低 60\%,结合全球 CDN 部署实现负载均衡,页面响应时间控制在 200ms 以内。
    \end{itemize}
  }[Vue3, Vite, NaiveUI, TailwindCSS, Pinia, ECharts]
  % \separator{0.5ex}
  % \project{2023-09}[2023-10]{独立开发}{\link{https://openinsider.pages.dev}{https://openinsider.pages.dev}}{OpenInsider+}{
  %   \begin{itemize}
  %     \item 美股持有人信息查询平台,支持订阅股票持有人变动通知,满足实时数据需求。
  %     \item 基于 CF Worker 构建 Serverless 服务,结合 KV 存储实现高效数据管理与低延迟响应。
  %     \item 集成 Cron 任务与 Telegram Bot,实现股票持有人变动的智能监控与实时推送。
  %   \end{itemize}
  % }[Vue3, Cheerio, Wrangler, Telegram Bot]
\end{projects}

%======================================================================
\sectionTitle{开源经历}{\faCode}
%======================================================================
\begin{projects}
  \project{2025-02}[至今]{200+ stars}{\link{https://github.com/luoling8192/tg-search}{https://github.com/luoling8192/tg-search}}{Telegram Search}{
    \begin{itemize}
      \item 基于 pgvector 向量数据库实现 Telegram 聊天记录的语义化搜索引擎,支持多维度相似度匹配与智能推荐,采用分表与分片策略优化数据结构,实现千万级会话数据的高效检索。
      \item 设计可扩展的 Adapter 架构,实现多账号及主流 IM 平台的无缝对接,支持多种主流 Embedding 模型接入,提供统一的数据处理与管理接口。
      \item 构建基于 WebSocket 的实时任务调度系统,实现任务优先级管理与并发控制,支持毫秒级进度反馈与状态同步。
    \end{itemize}
  }[PgVector, PostgreSQL]
  \separator{0.5ex}
  \project{2021-08}[2021-10]{3.4k+ stars}{\link{https://github.com/Icalingua-plus-plus/Icalingua-plus-plus}{https://github.com/Icalingua-plus-plus/Icalingua-plus-plus}}{Icalingua++}{
    \begin{itemize}
      \item 深度集成 QQ 协议,优化多媒体处理流程,基于 FFmpeg 实现高效音视频转码与实时预览。
      \item 设计高可用 Bridge 中间件,支持分布式部署与负载均衡,确保消息同步的可靠性与持久化。
      \item 构建可扩展插件生态,提供丰富 API 接口,支持自定义脚本、界面样式及主题定制。
    \end{itemize}
  }[Vue, Electron, SocketIO, FFmpeg]
\end{projects}

%======================================================================
\sectionTitle{其他项目}{\faCube}
%======================================================================
\begin{itemize}
  \item \textbf{Airi-Minecraft}(2025-01):开发基于大语言模型的智能 Minecraft 机器人,支持自然语言指令解析与玩家实时交互,实现任务规划与环境自适应。
    \link{https://github.com/moeru-ai/airi-minecraft}{https://github.com/moeru-ai/airi-minecraft}
  \item \textbf{ClassTools}(2022-05):设计并实现班级信息管理系统,集成高考倒计时、作业管理、课程表及值日生排班功能,提供动态壁纸展示界面。
    \link{https://github.com/luoling8192/ClassTools}{https://github.com/luoling8192/ClassTools}
\end{itemize}

%======================================================================
\sectionTitle{获奖经历}{\faTrophy}
%======================================================================
\begin{itemize}
  \item 2024-08:
    \link{https://www.bilibili.com/video/BV1WtHQewEu4}{AI.DEA 人工智能创新应用大赛 \textbf{一等奖}}
  \item 2024-07:
    \link{https://github.com/guiiai/guii-devtools}{AdvantureX 黑客松 \textbf{一等奖}}
  \item 2020-12:
    \link{https://www.noi.cn}{NOIP 全国青少年信息学奥林匹克竞赛联赛 \textbf{三等奖}}
\end{itemize}

\end{document}
