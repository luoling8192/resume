% Chinese version
\documentclass[zh]{resume}

% Adjust icon size(default: same size as the text)
\iconsize{\Large}

% File information shown at the footer of the last page
\fileinfo{%
  \faCopyright{} 2024, RainbowBird \hspace{0.5em}
  \creativecommons{by}{3.0} \hspace{0.5em}
  \faEdit{} \today
}

\name{洛灵}{}

\tagline{前端开发工程师}

% Supported shapes: circular(default), square
% \photo[<shape>]{<width>}{<filename>}

\profile{
  \email{me@luoling.moe}
  \github{luoling8192} \\
  \address{苏州}
  \birthday{2003-11}
  \icontext{\faWeixin}{@luoling8192}
}

\begin{document}
\makeheader

%======================================================================
% Summary & Objectives
%======================================================================
\sectionTitle{概述}{\faInfo}
{\onehalfspacing\hspace{2em}%
曾获全国青少年信息学奥林匹克竞赛联赛三等奖,能熟练阅读英文技术文档。
拥有 5 年以上项目开发经验,熟练使用 Linux 系统,掌握前端开发技术,积极参与开源项目,致力于开发创新技术解决方案。
关键词: NodeJS, Vue, React, Vite, TailwindCSS
\par}


%======================================================================
\sectionTitle{技能}{\faWrench}
%======================================================================
\begin{competences}
  \comptence{语言}{%
    TypeScript, JavaScript, C++, Python, PHP, Java, Go, CSharp
  }
  \comptence{前端}{%
    Vue, Vite, Pinia, TailwindCSS, React, Redux, jQuery, HTML/CSS
  }
  \comptence{后端}{%
    NodeJS, Cloudflare Worker, Electron, Express, MySQL, PostgreSQL, Redis
  }
  \comptence{工具}{%
    Linux, Git, ChatGPT(RAG), Bash, Docker, Kubernetes
  }
\end{competences}

%======================================================================
\sectionTitle{独立项目}{\faBriefcase}
%======================================================================
\begin{projects}
  \project{2024-02}[2024-05]{预览地址}{\link{https://portforward.zeroteam.top}{https://portforward.zeroteam.top}}{PortForwardGo}{
    \begin{itemize}
      \item 项目背景:开发了一个流量转发面板,旨在帮助用户便捷地管理和监控流量转发,项目日活用户超过 3,000。
      \item 技术细节:使用 Vue3 + Vite 进行前端开发,采用 TailwindCSS 作为 CSS 框架,并使用 Pinia 进行状态管理,实现了高效的前端架构。
      \item 组件系统:设计了高度抽象的组件系统,通过 JSON 配置文件生成复杂表单,保证了高灵活性,减少了开发和维护成本。
      \item 支付集成:实现了多种支付方式集成(如支付宝、Stripe等),提升了用户支付体验和平台的商业化能力。
      \item 实时监控:利用 WebSocket 实时监控服务器状态,及时响应异常情况,保证了系统的高可用性和稳定性。
      \item 数据可视化:使用 ECharts 进行数据可视化,提供详细的流量统计和分析图表,帮助用户直观了解流量使用情况。
    \end{itemize}
  }[Vue3, Vite, NaiveUI, TailwindCSS, Pinia, ECharts]

  \separator{0.5ex}
  \project{2023-09}[2023-10]{预览地址}{\link{https://openinsider.pages.dev}{https://openinsider.pages.dev}}{OpenInsider+}{
    \begin{itemize}
      \item 项目背景:开发了一个美股持有人查询平台,旨在提升用户对股票市场的透明度和分析能力。
      \item 功能实现:用户可以通过输入股票代码查询持有人变动信息,使用后端爬虫获取实时数据,并通过 Cheerio 提取和处理所需信息,确保了数据的准确性和实时性。
      \item 技术架构:通过 Cloudflare Worker 搭建 Serverless 后端服务,利用 KV 存储用户数据和交易记录,系统高效且可扩展。
      \item 筛选功能:设计了多个筛选功能,包括交易日期、价格区间和价值区间,帮助用户精确查找信息。
      \item 通知系统:使用 Cron 定时监测股票持有人变动,通过 Telegram Bot 发送异动通知,确保用户及时获取重要信息。
    \end{itemize}
  }[Vue3, Cheerio, Wrangler, Telegram Bot]

  \separator{0.5ex}
  \project{2021-07-11}[2021-07-16]{Github}{\link{https://github.com/luoling8192/lnry}{https://github.com/luoling8192/lnry}}{羊奶公司网站}{
    \begin{itemize}
      \item 项目背景:为羊奶公司开发了一个响应式网站,提升公司的线上形象和用户互动。
      \item 功能实现:提供新闻发布、图片展示和留言评论功能,使用 Umi.js 进行前端开发,后端采用 Leancloud 作为云数据库,实现数据存储和管理。
      \item 留言功能:增加用户留言和反馈功能,促进公司与用户之间的互动,提升客户满意度。
      \item 数据管理:利用 Leancloud 提供的 BaaS 服务,实现数据的高效管理和维护,提升了系统的可扩展性和可靠性。
    \end{itemize}
  }[React, UmiJS, Leancloud]

  \separator{0.5ex}
  \project{2020-03}[2020-05]{Github}{\link{https://github.com/luoling8192/Group-Website}{https://github.com/luoling8192/Group-Website}}{信息社团网站}{
    \begin{itemize}
      \item 项目背景:为信息社团开发了一个管理网站,增强社团的管理效率。
      \item 功能实现:实现了后台管理、用户登录注册、新闻发布、图片上传(OSS)等功能,使用 PHP 进行后端开发,前端采用 jQuery 实现动态交互。
      \item 访客统计:提供详细的访客统计功能,帮助社团了解网站访问情况,优化内容和功能。
    \end{itemize}
  }[PHP, jQuery]

\end{projects}

%======================================================================
\sectionTitle{开源经历}{\faCode}
%======================================================================
\begin{projects}
  \project{2022-05}[2023-04]{130+ stars}{主要开发人员}{ClassTools}{
    \begin{itemize}
      \item \link{https://github.com/clansty/ClassTools}{https://github.com/clansty/ClassTools}
      \item 项目背景:开发了一个用于班级电脑的教室系统,通过 Hook 系统的 Explorer 进程,实现动态壁纸功能,并已上架微软商店。
      \item 主要功能:提供高考倒计时、作业、课程表和值日生看板,具有简单易用的设置界面。
      \item 实时天气:对接和风天气 API 接口,可以显示 1-7 天的天气预报,并提供实时天气预警。
      \item 更新机制:实现了 Electron 自带更新器,确保用户可以方便地获取最新版本的功能和修复。
      \item 数据统计:通过发送统计日志实现日活统计,帮助开发团队了解用户活跃度和使用情况。
    \end{itemize}
  }[Vue3, Electron]

  \separator{0.5ex}
  \project{2021-08}[2021-10]{3.3k+ stars}{前端主要开发人员}{Icalingua++}{
    \begin{itemize}
      \item \link{https://github.com/Icalingua-plus-plus/Icalingua-plus-plus}{https://github.com/Icalingua-plus-plus/Icalingua-plus-plus}
      \item 项目背景:开发了一个第三方 QQ 客户端,使用 Electron 开发,通过对接 QQ 协议库,实现正常聊天以及仅有手机 QQ 才支持的戳一戳等功能,提升了用户的聊天体验。
      \item 主要功能:通过 FFmpeg 处理音视频,使用 SocketIO 实现前后端通信,对接 Aria2 实现群文件多线程下载,支持 MongoDB、MySQL、PostgreSQL 和 SQLite 作为聊天记录存储的数据库。
      \item 多平台支持:项目支持多平台运行,包括 Windows、Mac 和 Linux 系统,扩大了用户群体。
      \item 高度可定制性:支持运行自定义脚本、样式、主题、插件等,可以轻松定制和扩展功能。
      \item 数据导出:支持导出 QQ 聊天记录,并支持导出为 HTML 格式,方便用户进行离线浏览。
      \item Bridge 系统:本项目支持部署在服务器上运行,可以在任意客户端上对接使用,实现离线获取聊天记录功能。
    \end{itemize}
  }[Vue, Electron, SocketIO, FFmpeg]
\end{projects}

\end{document}
