% Chinese version
\documentclass[zh]{resume}

% Adjust icon size(default: same size as the text)
\iconsize{\Large}

% File information shown at the footer of the last page
\fileinfo{%
  \faCopyright{} 2024, RainbowBird \hspace{0.5em}
  \creativecommons{by}{3.0} \hspace{0.5em}
  \faEdit{} \today
}

\name{洛灵}{}

\keywords{NodeJS, Vue, React, Vite, TailwindCSS}

% \tagline{\icon{\faBinoculars}} <position-to-look-for>}
% \tagline{<current-position>}

% Supported shapes: circular(default), square
% \photo[<shape>]{<width>}{<filename>}

\profile{
  \mobile{177-6440-4156}
  \email{me@luoling.moe}
  \github{luoling8192} \\
  \icontext{\faBriefcase}{前端开发工程师}
  \birthday{2003-11}
  \address{苏州}
  % Custom information:
  % \icontext{<icon>}{<text>}
  % \iconlink{<icon>}{<link>}{<text>}
}

\begin{document}
\makeheader

%======================================================================
% Summary & Objectives
%======================================================================
{\onehalfspacing\hspace{2em}%

曾获全国青少年信息学奥林匹克竞赛联赛三等奖, 能熟练阅读英文技术文档。
拥有 5 年以上项目开发经验,熟练使用 Linux 系统,熟练掌握前端开发技术,积极参与开源项目,致力于开发创新技术解决方案。

\par}

%======================================================================
\sectionTitle{技能(按熟练度排序)}{\faWrench}
%======================================================================
\begin{competences}
  \comptence{语言}{%
    TypeScript, JavaScript, C++, Python, PHP, Java, Go, CSharp
  }
  \comptence{前端}{%
    Vue, Vite, Pinia, TailwindCSS, React, Redux, jQuery, HTML/CSS
  }
  \comptence{后端}{%
    NodeJS, Cloudflare Worker, Electron,Express, MySQL, PostgreSQL, Redis
  }
  \comptence{工具}{%
    Linux, Git, ChatGPT(RAG), Bash, Docker, Kubernetes
  }
\end{competences}

%======================================================================
\sectionTitle{独立项目}{\faBriefcase}
%======================================================================
\begin{projects}
  \project{2024-02}[2024-05]{预览地址}{\link{https://portforward.zeroteam.top}{https://portforward.zeroteam.top}}{PortForwardGo}{
    \begin{itemize}
      \item 流量转发面板,日活 3K+,除了实现基本用户管理员面板外,还实现了商店(对接多个支付平台)、服务器状态监控(Websocket)、流量统计图表(ECharts)等功能。
      \item 使用 Vue3 + Vite 开发,使用 TailwindCSS 作为 CSS 框架,并使用 Pinia 作为状态管理库。
      \item 高度抽象组件,所有复杂表单都可以通过 JSON 配置文件生成,且能保持相当高的灵活程度。
    \end{itemize}
  }[Vue3, Vite, NaiveUI, TailwindCSS, Pinia, ECharts]

  \separator{0.5ex}
  \project{2023-09}[2023-10]{预览地址}{\link{https://openinsider.pages.dev}{https://openinsider.pages.dev}}{OpenInsider+}{
    \begin{itemize}
      \item 美股持有人查询网站,使用后端爬虫获取页面,并用 Cheerio 提取所需数据。
      \item 通过 Cloudflare Worker 搭建 Severless 的后端服务,并使用 KV 存储用户、交易记录等数据。
      \item 通过 Cron 定时监测股票持有人变动,并通过 Telegram Bot 向对应用户发送信息。
    \end{itemize}
  }[Vue3, Cheerio, Wrangler, Telegram Bot]

  \separator{0.5ex}
  \project{2021-07-11}[2021-07-16]{Github}{\link{https://github.com/luoling8192/lnry}{https://github.com/luoling8192/lnry}}{羊奶公司网站}{
    \begin{itemize}
      \item 羊奶公司网站,采用响应式布局,使用 Umi.js 开发,后端使用 Leancloud 作为云数据库,并支持新闻(Markdown)、图片(OSS)、留言评论功能。
    \end{itemize}
  }[React, UmiJS, Leancloud]

  % \separator{0.5ex}
  % \project{2020-03}[2020-05]{Github}{\link{https://github.com/luoling8192/Group-Website}{https://github.com/luoling8192/Group-Website}}{信息社团网站}{
  %   \begin{itemize}
  %     \item 使用 PHP 编写,实现了后台管理面板,用户登录注册,查看发布新闻,图片上传(OSS),流量统计等功能。
  %   \end{itemize}
  % }[PHP, JavaScript]

\end{projects}

%======================================================================
\sectionTitle{开源经历}{\faCode}
%======================================================================
\begin{projects}
  % \project{2022-05}[2023-04]{130+ stars}{主要开发人员}{ClassTools}{
  %   \begin{itemize}
  %     \item \link{https://github.com/clansty/ClassTools}{https://github.com/clansty/ClassTools}
  %     \item 一个用在班级电脑上面的教室系统,通过 Hook 系统的 Explorer 进程,从而实现了动态壁纸功能,并已上架微软商店。
  %     \item 实现了高考倒计时、作业、课程表、值日生等看板,并提供简单易用的设置界面。
  %   \end{itemize}
  % }[Vue3, Electron]

  \separator{0.5ex}
  \project{2021-08}[2021-10]{3.3k+ stars}{前端主要开发人员}{Icalingua++}{
    \begin{itemize}
      \item \link{https://github.com/Icalingua-plus-plus/Icalingua-plus-plus}{https://github.com/Icalingua-plus-plus/Icalingua-plus-plus}
      \item 一个第三方 QQ 客户端,使用 Electron 开发,通过对接 QQ 协议库,不仅能够正常聊天,还实现了仅有手机 QQ 才支持的戳一戳等功能。
      \item 通过 FFmpeg 来处理音视频,并使用 SocketIO 来实现前后端通信。对接 Aria2 实现群文件多线程下载,支持 MongoDB、MySQL、PostgreSQL、SQLite 作为聊天记录存储的数据库。
    \end{itemize}
  }[Vue, Electron, SocketIO, FFmpeg]
\end{projects}

\end{document}
