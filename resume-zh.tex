% Chinese version
\documentclass[zh]{resume}

% Adjust icon size(default: same size as the text)
\iconsize{\Large}

% File information shown at the footer of the last page
\fileinfo{%
  \faCopyright{} 2025, RainbowBird \hspace{0.5em}
  \creativecommons{by}{3.0} \hspace{0.5em}
  \faEdit{} \today
}

\name{洛灵}{}

\tagline{Go 开发工程师 / 全栈开发工程师}

% Supported shapes: circular(default), square
% \photo[<shape>]{<width>}{<filename>}

\profile{
  \email{rbxin2003@outlook.com}
  \github{luoling8192} \\
  \address{远程}
  \birthday{2003-11}
  \icontext{\faWeixin}{@luoling8192}
}

\begin{document}
\makeheader

%======================================================================
% Summary & Objectives
%======================================================================
\sectionTitle{概述}{\faInfo}
{\onehalfspacing\hspace{2em}%
具备基本的区块链开发、LLM 方面相关知识,有 Golang、k8s 开发经验。
拥有 5 年以上前端开发经验, 1 年全栈开发经验,熟练掌握前端开发技术栈和 Linux 系统运维。
擅长使用函数式编程思想,可以流畅阅读英文文档,学习能力强,有多个开源项目贡献经历,致力追求最佳实践。
\par}


%======================================================================
\sectionTitle{技能}{\faWrench}
%======================================================================
\begin{competences}
  \comptence{\textbf{区块链开发}}{
    以太坊节点运维 \quad
    区块事件处理 \quad
    智能合约交互 \quad
    Web3 开发
  }
  
  \comptence{\textbf{后端开发}}{
    Go (1.20+) \quad
    微服务架构 \quad
    gRPC/REST API \quad
    分布式系统设计
  }
  
  \comptence{\textbf{技术栈}}{
    Echo · Fx · Ent \quad
    Kafka · Redis \quad
    PostgreSQL \quad
    TypeScript · Python
  }
  
  \comptence{\textbf{云原生}}{
    Kubernetes 集群管理 \quad
    Docker 容器化 \quad
    AWS 云服务 \quad
    CI/CD 自动化
  }
  
  \comptence{\textbf{可观测性}}{
    Prometheus 监控 \quad
    Grafana 看板 \quad
    OpenTelemetry 追踪 \quad
    ELK 日志分析
  }
  
  \comptence{\textbf{全栈开发}}{
    Vue3/React \quad
    Vite · Pinia \quad
    Serverless \quad
    Monorepo
  }
\end{competences}

%======================================================================
\sectionTitle{工作经历}{\faBriefcase}
%======================================================================
\begin{experiences}
  \experience%
    [2024-06]%
    {至今}%
    {区块链基础设施工程师 @ 八位猫(上海)}%
    [\begin{itemize}
      \item \textbf{区块链节点基础设施平台}
        \begin{itemize}
          \item 设计并实现高可用区块链节点集群,支持 Ethereum、Astar 等多链架构。
          \item 基于 K8s 实现节点服务编排与自动化运维,提升部署效率 80\%。
          \item 构建完整监控告警体系,实现故障 5 分钟内定位修复,系统可用性达 99.9\%。
          \item 开发区块数据并行处理工具,数据同步性能提升 3 倍。
        \end{itemize}
      \item \textbf{高性能区块链事件处理系统}
        \begin{itemize}
          \item 主导设计微服务架构,实现交易、区块等多类型事件处理,日均处理 100 万笔交易。
          \item 采用 Fx 依赖注入 + Echo 微服务 + Kafka 消息队列构建分布式系统。
          \item 通过连接池优化与分布式事务,实现 5k+ QPS 稳定处理,响应时间 < 50ms。
          \item 基于 OpenTelemetry 实现全链路监控,问题定位效率提升 60\%。
          \item 设计数据分片与缓存策略,查询性能提升 5 倍,内存使用降低 40\%。
          \item 实现核心模块 95\% 测试覆盖率,部署后故障率降低 70\%。
        \end{itemize}
      \end{itemize}]
\end{experiences}

%======================================================================
\sectionTitle{项目经历}{\faBriefcase}
%======================================================================
\begin{projects}
  \project{2024-07}[2024-12]{后端负责人}{\link{https://guii.ai}{https://guii.ai}}{Guii.AI}{
    \begin{itemize}
      \item 一款 AI 赋能的前端开发工具,覆盖从原型设计到细节调优的全流程,支持通过自然语言直接生成和调整界面。
      \item 创新性地实现页面元素直接选择和编辑功能,无需 IDE 集成即可灵活融入现有开发工具链。
      \item 在 2024 年 AdventureX 黑客松中斩获主题赛道一等奖及其他三个赛道一等奖。
      \item 作为后端核心开发者,负责提示词引擎、工作流系统和 Agent 编排的设计与实现。
    \end{itemize}
  }[HonoJS, Supabase, PostgreSQL, Redis, Drizzle, OpenRouter]

  \separator{0.5ex}
  \project{2024-02}[2024-05]{前端负责人}{\link{https://portforward.zeroteam.top}{https://portforward.zeroteam.top}}{PortForwardGo}{
    \begin{itemize}
      \item 开发高性能流量转发管理平台,支持多协议转发和负载均衡,日活跃用户超过 3000。
      \item 设计模块化组件系统,实现可配置的动态界面生成,提升开发效率 70\%,大幅降低维护成本。
      \item 基于 Vue3 + Vite 构建现代化前端架构,实现状态管理与组件复用,集成多渠道支付系统和实时数据可视化面板。
    \end{itemize}
  }[Vue3, Vite, NaiveUI, TailwindCSS, Pinia, ECharts]

  \separator{0.5ex}
  \project{2023-09}[2023-10]{独立开发}{\link{https://openinsider.pages.dev}{https://openinsider.pages.dev}}{OpenInsider+}{
    \begin{itemize}
      \item 美股持有人信息查询平台,支持订阅股票持有人变动通知。
      \item 基于 Cloudflare Worker 构建 Serverless 服务,配合 KV 存储实现高性能数据管理。
      \item 结合 Cron 任务和 Telegram Bot,实现股票持有人变动的实时监控和智能提醒。
    \end{itemize}
  }[Vue3, Cheerio, Wrangler, Telegram Bot]
\end{projects}

%======================================================================
\sectionTitle{开源经历}{\faCode}
%======================================================================
\begin{projects}
  \project{2025-02}[至今]{200+ stars}{独立开发}{Telegram Search}{
    \begin{itemize}
      \item \link{https://github.com/luoling8192/tg-search}{https://github.com/luoling8192/tg-search}
      \item 基于 pgvector 向量数据库实现分表存储,支持多种 Embedding 模型的语义检索。
      \item 设计 Adapter 架构实现多账号 / IM 平台对接,支持灵活扩展和统一管理。
      \item 实现动态表分片和自动扩容,单表支持千万级向量数据的高性能检索。
    \end{itemize}
  }[PgVector, PostgreSQL]

  \separator{0.5ex}
  \project{2021-08}[2021-10]{3.4k+ stars}{前端核心开发}{Icalingua++}{
    \begin{itemize}
      \item \link{https://github.com/Icalingua-plus-plus/Icalingua-plus-plus}{https://github.com/Icalingua-plus-plus/Icalingua-plus-plus}
      \item 深度集成 QQ 协议,优化多媒体处理流程,基于 FFmpeg 实现高性能音视频转码与实时预览。
      \item 设计高可用的 Bridge 中间件系统,支持分布式部署与负载均衡,实现消息可靠同步与持久化存储。
      \item 构建可扩展的插件生态系统,提供丰富的 API 接口,支持自定义脚本开发、界面样式定制和主题系统。
    \end{itemize}
  }[Vue, Electron, SocketIO, FFmpeg]
\end{projects}

%======================================================================
\sectionTitle{其他项目}{\faCube}
%======================================================================
\begin{itemize}
  \item 开发基于大语言模型的智能 Minecraft 机器人,实现自然语言指令理解与玩家实时交互,支持任务规划和环境适应。
    \link{https://github.com/moeru-ai/airi-minecraft}{\texttt{Airi-Minecraft 2025-01}}
  \item 设计并实现班级信息管理系统,集成高考倒计时、作业管理、课程表和值日生排班等功能,提供动态壁纸展示界面。
    \link{https://github.com/luoling8192/ClassTools}{\texttt{ClassTools 2022-05}}
\end{itemize}

%======================================================================
\sectionTitle{获奖经历}{\faTrophy}
%======================================================================
\begin{itemize}
  \item 2024-08
    \link{https://www.bilibili.com/video/BV1WtHQewEu4}{\texttt{AI.DEA 人工智能创新应用大赛 \textbf{一等奖}}}
  \item 2024-07
    \link{https://github.com/guiiai/guii-devtools}{\texttt{AdvantureX 黑客松 \textbf{一等奖}}}
  \item 2020-12
    \link{https://www.noi.cn}{\texttt{NOIP 全国青少年信息学奥林匹克竞赛联赛 \textbf{三等奖}}}
\end{itemize}

\end{document}
