% Chinese version
\documentclass[zh]{resume}

% Adjust icon size (default: same size as the text)
\iconsize{\Large}

% File information shown at the footer of the last page
\fileinfo{%
  \faCopyright{} 2024, RainbowBird \hspace{0.5em}
  \creativecommons{by}{3.0} \hspace{0.5em}
  \faEdit{} \today
}

% \name{方鑫}{刘}
\name{洛灵}{}

\keywords{NodeJS, Vue, React, Vite, TailwindCSS}

% \tagline{\icon{\faBinoculars}} <position-to-look-for>}
% \tagline{<current-position>}

% Supported shapes: circular (default), square
% \photo[<shape>]{<width>}{<filename>}

\profile{
  \mobile{177-6440-4156}
  \email{me@luoling.moe}
  \github{luoling8192} \\
  % \university{高中毕业}
  % \degree{物理学 \textbullet 博士}
  \birthday{2003-11}
  \address{苏州}
  % Custom information:
  % \icontext{<icon>}{<text>}
  % \iconlink{<icon>}{<link>}{<text>}
}

\begin{document}
\makeheader

%======================================================================
% Summary & Objectives
%======================================================================
{\onehalfspacing\hspace{2em}%

小学开始学习编程,高中时获得全国青少年信息学奥林匹克竞赛联赛三等奖,能熟练阅读英文技术文档。\\
具有强烈的责任心和优秀的团队合作精神,对计算机科学充满热情,喜欢研究最新技术并追求最佳实现方案。\\
拥有 5 年以上项目开发经验,熟练使用 Linux 系统,熟练掌握前端开发技术,积极参与开源项目,致力于开发创新技术解决方案。\\
为 \link{https://github.com/Icalingua-plus-plus/Icalingua-plus-plus}{Icalingua++} , \link{https://github.com/IcariaWorks/ClassTools}{ClassTools+} 等开源项目贡献过代码。

% \link{https://github.com/luoling8192}{Github}
\par}

%======================================================================
\sectionTitle{技能}{\faWrench}
%======================================================================
\begin{competences}
  % \comptence{操作系统}{%
  %   \icon{\faLinux} Linux (10 年),
  %   \icon{\faFreebsd} DragonFly BSD \& FreeBSD (7 年)
  % }
  \comptence{语言}{%
    TypeScript, JavaScript, C++, Python, PHP, Java, Go, CSharp
  }
  \comptence{前端}{%
    Vue, Vite, Pinia, TailwindCSS, React, Redux, jQuery, HTML/CSS
  }
  \comptence{后端}{%
    NodeJS, Express, Cloudflare Worker, MySQL, PostgreSQL, Redis
  }
  \comptence{工具}{%
    Linux, Git, Bash, ChatGPT(RAG), Docker, Kubernetes
  }
  % \comptence{\icon{\faLanguage} 语言}{
  %   \textbf{英语} --- 读写(优良),听说(日常交流)
  % }
\end{competences}

% %======================================================================
% \sectionTitle{教育背景}{\faGraduationCap}
% %======================================================================
% \begin{educations}
%   \education%
%     {2013.09}%
%     [2019.09]%
%     {上海交通大学}%
%     {物理与天文学院}%
%     {物理学}%
%     {博士}

%   \separator{0.5ex}
%   \education%
%     {2009.09}%
%     [2013.06]%
%     {上海交通大学}%
%     {物理与天文系}%
%     {应用物理学}%
%     {学士}
% \end{educations}

% %======================================================================
% \sectionTitle{计算机技能}{\faCogs}
% %======================================================================
% \begin{itemize}
%   \item DragonFly BSD 操作系统开发者:
%     200+ 代码提交;内核以及系统工具;
%     在邮件列表和 IRC 频道交流和回答问题
%   \item 使用 Ansible 管理 VPS,部署个人域名邮箱、权威 DNS、网站、Git、IRC 等服务
%   \item 搭建并管理课题组的工作站、计算集群(4 节点)和网络设备
%   \item 参与配置和测试上海天文台的 SKA 高性能计算集群原型机
%     (1 管理节点 + 1 存储节点 + 4 计算节点)
%   \item 设计并开发了\enquote{2014 第一届中国—新西兰联合 SKA 暑期学校}的整个网站
%     (Django, Bootstrap, jQuery)
% \end{itemize}

%======================================================================
\sectionTitle{项目经历}{\faCode}
%======================================================================
\begin{itemize}
  \item \texttt{PortForwardGo 2024-02}:
    (Vue3, NaiveUI, Vite, TailwindCSS, Pinia, TypeScript)\\
    \link{https://portforward.zeroteam.top/}{预览地址}\\
    一个流量转发面板,支持多用户管理,账单管理,内网穿透主机管理等功能,
    模块化开发,实现了动态表单,所有页面都可以通过一个 JSON 配置生成相应表单,能够最大程度的复用组件。\\
  \item \texttt{OpenInsider+ 2023-08}:
    (Vue3, NaiveUI, Vite, Wrangler, TypeScript)\\
    \link{https://openinsider.pages.dev/}{预览地址}\\
    美股股票持有人查询网站,基于爬虫获取页面数据,使用 Cloudflare Worker 搭建 Severless 的后端服务。\\
    检测股票持有人变动,并通过 Telegram Bot 向对应用户发送信息。\\
  \item \texttt{ClassTools 2021-12}:
    (Express, Antd, Axios, React, TypeScript)\\
    \link{https://github.com/luoling8192/ClassTools}{Github}
    \link{https://github.com/luoling8192/ClassTools-Backend}{后端地址}\\
    一个简单的班级管理系统(看板),作为系统壁纸展示,数据存储在本地,前后端分离。\\
    实现功能:作业布置、天气显示(和风天气API)、高考倒计时、壁纸更换。\\
  \item \texttt{羊奶公司宣传网站 2021-07}:
    (React, Umi, Antd, Leancloud, jQuery, TypeScript, Less, HTML5)\\
    \link{https://github.com/luoling8192/lnry}{Github}
    \link{https://lnry.pages.dev/}{预览地址}\\
    羊奶公司宣传网站,设计精美简练,符合商业公司宣传理念,采用响应式布局。\\
    实现功能:产品展示(图片轮播)、企业资讯(Markdown、评论区)、加入我们(表单、验证码)。\\
  \item \texttt{高中信息社团网站 2020-05}:
    (PHP, JavaScript, HTML5, CSS)\\
    \link{https://github.com/luoling8192/Group-Website}{Github}
    \link{https://xjez.pages.dev}{预览地址}\\
    使用 PHP 编写,实现了后台管理面板,用户登录注册,查看发布新闻,图片上传(OSS),流量统计。
\end{itemize}

%======================================================================
\sectionTitle{个人项目}{\faCube}
%======================================================================
\begin{itemize}
  \item 基于水鱼查分器的 DXRating 生成器,可以获取 Rating 分数以及生成 Rating 图片。
    \link{https://github.com/luoling8192/dxrating}{\texttt{DxRating 2024-01}}
  \item Chrome 插件,通过 ChatGPT API 实现对书签页自动重命名。
    \link{https://github.com/luoling8192/chrome-ai-remark}{\texttt{Chrome-AI-Remark 2023-12}}
\end{itemize}

% %======================================================================
% \sectionTitle{实习经历}{\faBriefcase}
% %======================================================================
% \begin{experiences}
%   \experience%
%     [2018.04]%
%     {2018.08}%
%     {数据工程师 @ 上海领脉网络科技(初创公司)}%
%     [\begin{itemize}
%       \item 从 Amazon 网页搜索并挖取商品与广告信息
%         (Python, Requests, BeautifulSoup)
%       \item 配置 Airflow 服务器和数据库等基础设施,
%         定期从 Amazon 获取产品销售与广告投放等数据
%       \item 开发网站(Flask, jQuery),帮助客户优化 Amazon 广告投放
%     \end{itemize}]

%   \separator{0.5ex}
%   \experience%
%     [2013.07]%
%     {2013.09}%
%     {网站开发 @ 97 随访(初创公司)}%
%     [\begin{itemize}
%       \item 后端开发(Django),完成用户注册、数据存储和搜索等功能
%       \item 前端开发(jQuery, AJAX),对患者各项指标随时间的变化进行可视化
%     \end{itemize}]
% \end{experiences}

\end{document}
