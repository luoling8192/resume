% Chinese version
\documentclass[zh]{resume}

% Adjust icon size(default: same size as the text)
\iconsize{\Large}

% File information shown at the footer of the last page
\fileinfo{%
  \faCopyright{} 2025, RainbowBird \hspace{0.5em}
  \creativecommons{by}{3.0} \hspace{0.5em}
  \faEdit{} \today
}

\name{洛灵}{}

\tagline{Go 开发工程师 / 全栈开发工程师}

% Supported shapes: circular(default), square
% \photo[<shape>]{<width>}{<filename>}

\profile{
  \email{rbxin2003@outlook.com}
  \github{luoling8192} \\
  \address{远程}
  \birthday{2003-11}
  \icontext{\faWeixin}{@luoling8192}
}

\begin{document}
\makeheader

%======================================================================
% Summary & Objectives
%======================================================================
\sectionTitle{概述}{\faInfo}
{\onehalfspacing\hspace{2em}%
具备基本的区块链开发、LLM 方面相关知识,有 Golang、k8s 开发经验。
拥有 5 年以上前端开发经验, 1 年全栈开发经验,熟练掌握前端开发技术栈和 Linux 系统运维。
擅长使用函数式编程思想,可以流畅阅读英文文档,学习能力强,有多个开源项目贡献经历,致力追求最佳实践。
\par}


%======================================================================
\sectionTitle{技能}{\faWrench}
%======================================================================
\begin{competences}
  \comptence{\textbf{区块链运维}}{
    AWS \quad
    Ethereum 节点集群 \quad
    Kubernetes 编排 \quad
    Prometheus / Grafana 监控体系
  }
  \comptence{\textbf{Go 开发}}{
    Echo (gRPC/REST) \quad
    Ent ORM \quad
    Fx \quad
    Kafka \quad
    PostgreSQL \quad
    Redis
  }
  \comptence{\textbf{云原生}}{
    Docker 容器化 \quad
    Kubernetes 运维 \quad
    CI/CD 流水线 \quad
    OpenTelemetry 可观测性
  }
  \comptence{编程语言}{
    TypeScript / NodeJS \quad
    Go (1.20+) \quad
    Python \quad
    SQL
  }
  \comptence{后端框架}{
    Express \quad
    HonoJS \quad
    Drizzle \quad
    Supabase \quad
    Serverless
  }
  \comptence{前端框架}{
    Vue3 \quad
    React \quad
    Vite \quad
    UnoCSS \quad
    Pinia \quad
    Electron \quad
    Monorepo
  }
\end{competences}

%======================================================================
\sectionTitle{工作经历}{\faBriefcase}
%======================================================================
\begin{experiences}
  \experience%
    [2024-06]%
    {至今}%
    {区块链运维 / Go 开发工程师 @ 八位猫(上海)}%
    [\begin{itemize}
      \item \textbf{区块链基础设施}
        \begin{itemize}
          \item 使用 AWS EC2 部署和维护 Ethereum / Astar 等节点集群。
          \item 基于 k8s (Pod / ConfigMap / Service)编排节点服务,实现滚动更新和自动扩缩容。
          \item 搭建 Prometheus 监控告警体系,构建 Grafana 数据看板,实现节点健康状态实时监控。
          \item 开发高并发区块链索引工具,采用 Goroutine 池处理区块数据
        \end{itemize}
      \item \textbf{区块链交易事件处理系统}
        \begin{itemize}
          \item 处理 Ethereum 主网交易、区块更新、转账等事件。
          \item 使用 Fx 进行依赖注入、生命周期管理。
          \item 使用 Ent 连接 PostgreSQL,使用复合索引优化查询效率。
          \item 使用 Echo 实现基于 gRPC / Protobuf 的微服务通信和 Restful 网关。
          \item 使用 Kafka 实现消息队列,并使用 Redis 缓存热点数据。
          \item 使用 OpenTelemetry 实现链路追踪,并使用 Prometheus 监控告警。
          \item 使用分布式事务处理,解决并发冲突、保证幂等性。
          \item 使用 decimal 实现高精度数值计算,保障金融级计算准确性。
          \item 通过连接池优化实现 5k+ QPS 的稳定处理能力。
          \item 实现核心模块单元测试 95\% 覆盖率。
        \end{itemize}
      \end{itemize}]
\end{experiences}

%======================================================================
\sectionTitle{项目经历}{\faBriefcase}
%======================================================================
\begin{projects}
  \project{2024-07}[2024-12]{后端核心开发者}{\link{https://guii.ai}{https://guii.ai}}{Guii.AI}{
    \begin{itemize}
      \item 一款 AI 赋能的前端开发结对编程工具,帮助开发者通过自然语言直接生成可视化的前端界面。
      \item 在 2024 年 AdventureX 黑客松中斩获主题赛道一等奖及其他三个赛道一等奖。
      \item 作为后端核心开发者,负责提示词,工作流和 Agent 编排的开发。
    \end{itemize}
  }[HonoJS, Supabase, PostgreSQL, Redis, Drizzle, OpenRouter]

  \separator{0.5ex}
  \project{2024-02}[2024-05]{独立开发}{\link{https://portforward.zeroteam.top}{https://portforward.zeroteam.top}}{PortForwardGo}{
    \begin{itemize}
      \item 流量转发管理平台,日活跃用户 3k+。
      \item 设计高度抽象的组件系统,实现灵活的界面生成,降低 70\% 开发维护成本。
      \item 采用 Vue3 + Vite 前端架构,实现组件化开发和状态管理,整合多支付渠道和数据可视化。
    \end{itemize}
  }[Vue3, Vite, NaiveUI, TailwindCSS, Pinia, ECharts]

  \separator{0.5ex}
  \project{2023-09}[2023-10]{独立开发}{\link{https://openinsider.pages.dev}{https://openinsider.pages.dev}}{OpenInsider+}{
    \begin{itemize}
      \item 美股持有人信息查询平台,支持订阅股票持有人变动通知。
      \item 基于 Cloudflare Worker 构建 Serverless 服务,配合 KV 存储实现高性能数据管理。
      \item 结合 Cron 任务和 Telegram Bot,实现股票持有人变动的实时监控和智能提醒。
    \end{itemize}
  }[Vue3, Cheerio, Wrangler, Telegram Bot]
\end{projects}

%======================================================================
\sectionTitle{开源经历}{\faCode}
%======================================================================
\begin{projects}
  \project{2025-02}[至今]{100+ stars}{个人项目}{Telegram Search}{
    \begin{itemize}
      \item \link{https://github.com/luoling8192/tg-search}{https://github.com/luoling8192/tg-search}
      \item 基于 OpenAI 向量技术实现语义搜索,支持多种消息类型的智能检索。
      \item 设计高性能数据同步系统,支持 Bot / Client 双模式消息采集和增量更新。
      \item 实现灵活的消息管理体系,支持文件夹分类、导入导出和数据统计分析。
    \end{itemize}
  }[PgVector, PostgreSQL]

  \separator{0.5ex}
  \project{2021-08}[2021-10]{3.4k+ stars}{前端核心开发者}{Icalingua++}{
    \begin{itemize}
      \item \link{https://github.com/Icalingua-plus-plus/Icalingua-plus-plus}{https://github.com/Icalingua-plus-plus/Icalingua-plus-plus}
      \item 深度集成 QQ 协议,采用 FFmpeg 处理多媒体,SocketIO 实现实时通信。
      \item 设计 Bridge 系统支持服务器部署,实现多端同步和离线消息存储。
      \item 实现灵活的插件架构,支持自定义脚本、样式和主题。
    \end{itemize}
  }[Vue, Electron, SocketIO, FFmpeg]
\end{projects}

%======================================================================
\sectionTitle{其他项目}{\faCube}
%======================================================================
\begin{itemize}
  \item 基于 LLM 的智能 Minecraft 机器人,能理解自然语言指令并与玩家交互。
    \link{https://github.com/moeru-ai/airi-minecraft}{\texttt{Airi-Minecraft 2025-01}}
  \item 一个用在班级电脑上面的动态壁纸系统,显示高考倒计时、作业、课程表、值日生等看板。
    \link{https://github.com/luoling8192/ClassTools}{\texttt{ClassTools 2022-05}}
\end{itemize}

%======================================================================
\sectionTitle{获奖经历}{\faTrophy}
%======================================================================
\begin{itemize}
  \item 2024-08
    \link{https://www.bilibili.com/video/BV1WtHQewEu4}{\texttt{AI.DEA \textbf{一等奖}}}
  \item 2024-07
    \link{https://github.com/guiiai/guii-devtools}{\texttt{AdvantureX 主题赛道 \textbf{一等奖}}}
  \item 2020-12
    \link{https://www.noi.cn}{\texttt{全国青少年信息学奥林匹克竞赛联赛 \textbf{三等奖}}}
\end{itemize}

\end{document}
